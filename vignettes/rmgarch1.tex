\section{Introduction}
The ability to dynamically and jointly model the full multivariate density
dynamics has very important implications for risk and portfolio management,
and more generally economic policy decision making. However, feasible large-scale
multivariate GARCH modelling has proved very challenging since the direct extension
of the univariate models to a vector representation by \cite{Bollerslev1988}.
The {\bf rmgarch} package aims to provide a subset of multivariate GARCH models
which can handle large scale estimation through separation of the dynamics
so that parallel processing may be used. Methods for fitting, filtering,
forecasting and simulation are included were applicable with some interesting
additional methods aimed at portfolio and risk applications. This document provides
for a summarized theoretical background of the models and their properties.\\
While there are a number of open source and commercial packages implementing the
DCC based models, the {\bf rmgarch} package uniquely implements and introduces
the GO-GARCH model with ICA using the multivariate affine Generalized Hyperbolic
distribution and the relevant methods for working with this model in an applied setting.\\
The {\bf rmgarch} package is on CRAN and the development version on 
bitbucket (\url{https://bitbucket.org/alexiosg}). Some online examples and demos
are available on my website (\url{http://www.unstarched.net}).\\
The package is provided AS IS, without any implied warranty as to its accuracy
or suitability. A lot of time and effort has gone into the development of this
package, and it is offered under the GPL-3 license in the spirit of open knowledge
sharing and dissemination. If you do use the model in published work DO remember
to cite the package and author (type {\bf citation("rmgarch")} for the
appropriate BibTeX entry) , and if you have used it and found it useful, drop me
a note and let me know.\\
\textbf{ IMPORTANT :\\
The package is still in development and some functions/methods MAY change over time, 
and bugs are certain to exist. Please report any suspected bugs in the code, mistakes 
in the models or general questions to the R-SIG-FINANCE mailing list and not directly 
to my email, unless solicited. I maintain a blog (\url{http://www.unstarched.net}) 
which contains some examples and posts which I update when time permits.}
\section{Multivariate GARCH Models}
The generalization of univariate GARCH models to the multivariate domain is
conceptually simple. Consider the stochastic vector process, $\bfx_t$ $\{t=1,2,...T\}$
of financial returns with dimension $N\times 1$ and mean vector $\bfmu_t$\footnote{The mean
vector may for example be derived from a VAR model or may simply represent the
unconditional means of the financial returns.}, given the information set $\bfI_{t-1}$:
\begin{equation}\label{eq:mgarch1}
\bfx_t\left| \bfI_{t - 1} \right. = \bfmu _t + \bfvarepsilon _t,
\end{equation}
where the residuals of the process are modelled as:
\begin{equation}\label{eq:mgarch2}
{\bfvarepsilon _t} = \bfH_t^{1/2}{\bfz_t},
\end{equation}
and $\bfH_t^{1/2}$ is an $N\times N$ positive definite matrix such that $\bfH_t$
is the conditional covariance matrix of $\bfx_t$\footnote{One way to obtain the
square root matrix is through the singular value decomposition of $\bfH_t$.},
and $\bfz_t$ an $N\times 1$ i.i.d. random vector, with centered and scaled first 2 moments:
\begin{equation}\notag
\E\left[\bfz_t\right] = 0,
\end{equation}
\begin{equation}\label{eq:mgarch3}
\Var[{\bfz_t}] = {\bfI_N},
\end{equation}
with $\bfI_N$ denoting the identity matrix of order N. The conditional covariance
matrix $\bfH_t$ of $\bfx_t$ may be defined as:
\begin{align}
\Var\left(\bfx_t\left| \bfI_{t - 1}\right.\right) =  \Var_{t - 1}(\bfx_t) &= \Var_{t - 1}(\bfvarepsilon_t)\nonumber \\
& = \bfH_t^{1/2}\Var_{t - 1}(\bfz_t)(\bfH_t^{1/2})' \nonumber \\
& = \bfH_t.\label{eq:mgarch4}
\end{align}
The literature on the different specifications of $\bfH_t$ may be broadly divided
into direct multivariate extensions, factor models and the conditional correlation
models. The usual trade-off of model parametrization and dimensionality clearly
applies here, with the fully parameterized models offering the richest dynamics
at the cost of increasing parameter size, making it unfeasible for modelling
anything beyond a couple of assets. There is, also, a not so evident tradeoff
between those models which allow flexible univariate dynamics (in the motion
dynamics and the distributions) to enter the equation at the cost of some
multivariate dynamics. The next sections will review these models and some of
the tradeoffs they present for the decision maker. A more complete review of
multivariate GARCH (\emph{MGARCH}) models is provided by \cite{Bauwens2006}
and \cite{Silvennoinen2009a}.

\subsection{Conditional Mean Dynamics}\label{section:mmean}
The {\bf rmgarch} package allows for either a constant, univariate AR or
Vector AR (\emph{VAR}) model to be fit (or a pre-filtered residual series).
The constant and AR models are already implemented and described in the {\bf {rugarch}}
package. For the DCC based models, the constant-AR model is jointly estimated
with the first stage GARCH dynamics, while for the GO-GARCH models the univariate
ARFIMAX model is used assuming constant variance to obtain the parameter estimates.
In the case of the VAR model, external regressors are also allowed as is the
possibility to use a robust version of the model based on the multivariate least
trimmed squares approach of \cite{Croux2008}. When using a constant or AR model
with DCC based models, standard errors are calculated for all first stage
parameters using a partitioned standard error matrix. In the case of a VAR model,
this joint estimation of standard errors is not practical due to the dimensionality
of the system. Finally, in the case of the GO-GARCH model, there is no joint
estimation of parameters for the first (conditional mean) and second (factor dynamics)
stage estimation.\footnote{
Additionally, since ICA is a linear noiseless model, there is no uncertainty assumed
with regards to the mixing matrix A.}
\subsection{Dynamic Conditional Correlation Models}\label{section:dcc}
Conditional correlation models are founded on a decomposition of the conditional
covariance matrix into conditional standard deviations and correlations, so that
it may be expressed in such a way that the univariate and multivariate dynamics
may be separated, thus easing the estimation process. This decomposition comes
at a cost of some dynamic structure as well as severe restriction on the type
of multivariate distribution which can usually be decomposed in such a way.\footnote{
This has implications both for the use of the 2-stage estimation as well as the
form of the covariance matrix which may be a complicated function of the scaling
matrix for non-elliptical distributions.} Recently, some of these models have
been extended to allow for more flexible dynamic structure which unfortunately
has led to significant loss in the ease of estimation.\\
In the constant conditional correlation model (CCC) of \cite{Bollerslev1990},
the covariance matrix can be decomposed into
\begin{equation}\label{eq:dcc1}
{\bfH_t} = {\bfD_t}\bfR{\bfD_t} = {\rho _{ij}}\sqrt {{h_{iit}}{h_{jjt}}},
\end{equation}
where ${\bfD_t} = diag(\sqrt {{h_{11,t}}} ,...,\sqrt {{h_{nn,t}}} )$,
and $\bfR$ is the positive definite constant conditional correlation matrix.
The conditional variances, and $h_{ii,t}$, which can be estimated separately,
can be written in vector form based on GARCH(p,q) models\footnote{The GARCH
models are not restricted to be of one particular 'flavor', allowing to mix
different GARCH models in the univariate stage.}
\begin{equation}\label{eq:dcc2}
{h_t} = \omega  + \sum\limits_{i = 1}^p {{\bfA_i}{\varepsilon _{t - i}} \odot {\varepsilon _{t - i}} + } \sum\limits_{i = 1}^q {{\bfB_i}{h_{t - i}}}
\end{equation}
where $\omega  \in {\mathbb{R}^n}$, $\bfA_i$ and $\bfB_i$ are $N\times N$
diagonal matrices, and $\odot$ denotes the Hadamard operator. The conditions for
the positivity of the covariance matrix $\bfH_t$ are that $\bfR$ is positive
definite, and the elements of $\omega$ and the diagonal elements of the matrices
$\bfA_i$ and $\bfB_i$ are positive. In the extended CCC model (E-CCC) of
\cite{Jeantheau1998}, implemented in the {\bf ccgarch} package, the
assumption of diagonal elements on $\bfA_i$ and $\bfB_i$ was relaxed, allowing
the past squared errors and variances of the series to affect the dynamics of
the individual conditional variances, and hence providing for a much richer structure,
albeit at the cost of a lot more parameters. The decomposition in \eqref{eq:dcc1},
allows the log-likelihood at each point in time ($LL_t$), in the multivariate
normal case, to be expressed as
\begin{equation}\label{eq:dcc3}
\begin{gathered}
  {LL_t} = \frac{1}
{2}\left( {\log \left( {2\pi } \right) + \log \left| {{\bfH_t}} \right| + {{\bfvarepsilon '}_t}\bfH_t^{ - 1}{\bfvarepsilon _t}} \right) \hfill \\
   = \frac{1}
{2}\left( {\log \left( {2\pi } \right) + \log \left| {{\bfD_t}{\bfR}{\bfD_t}} \right| + {{\bfvarepsilon '}_t}\bfD_t^{ - 1}\bfR^{ - 1}\bfD_t^{ - 1}{\bfvarepsilon _t}} \right) \hfill \\
   = \frac{1}
{2}\left( {\log \left( {2\pi } \right) + 2\log \left| {{\bfD_t}} \right| + \log \left| {{\bfR}} \right| + {{\bfz'}_t}\bfR^{ - 1}{{\bfz'}_t}} \right) \hfill \\
\end{gathered}
\end{equation}
where ${\bfz_t} = \bfD_t^{ - 1}{\bfvarepsilon _t}$. This can be described as a
term ($\bfD_t$) for the sum of the univariate GARCH model likelihoods, a term
for the correlation ($\bfR$) and a term for the covariance which arises from
the decomposition.\\
Because the restriction of constant conditional correlation may be unrealistic
in practice, a class of models termed Dynamic Conditional Correlation (DCC) due
to \cite{Engle2002} and \cite{Tse2002a} where introduced which allow
for the correlation matrix to be time varying with motion dynamics, such that
\begin{equation}\label{eq:dcc4}
{\bfH_t} = {\bfD_t}{\bfR_t}{\bfD_t}.
\end{equation}
In these models, apart from the fact that the time varying correlation matrix,
$\bfR_t$, must be inverted at every point in time (making the calculation that
much slower), it is also important to constrain it to be positive definite.
The most popular of these DCC models, due to \cite{Engle2002}, achieves
this constraint by modelling a proxy process, $\bfQ_t$ as:
\begin{equation}\label{eq:dcc5}
\begin{gathered}
  {\bfQ_t} = \bar \bfQ + a\left( {{\bfz_{t - 1}}{{\bfz'}_{t - 1}} - \bar \bfQ} \right) + b\left( {{\bfQ_{t - 1}} - \bar \bfQ} \right) \hfill \\
        = (1 - a - b)\bar \bfQ + a{\bfz_{t - 1}}{{\bfz'}_{t - 1}} + b{\bfQ_{t - 1}} \hfill \\
\end{gathered}
\end{equation}
where $a$ and $b$ are non negative scalars, with the condition that $a + b < 1$
imposed to ensure stationarity and positive definiteness of $\bfQ_t$. $\bar Q$
is the unconditional matrix of the standardized errors $\bfz_t$ which enters the
equation via the covariance targeting part $(1-a-b)\bar \bfQ$, and $\bfQ_0$ is
positive definite. The correlation matrix $\bfR$ is then obtained by rescaling
$\bfQ_t$ such that,
\begin{equation}\label{eq:dcc6}
{\bfR_t} = diag{({\bfQ_t})^{ - 1/2}}{\bfQ_t}diag{({\bfQ_t})^{ - 1/2}}.
\end{equation}
The log-likelihood function in equation \eqref{eq:dcc2} can be
decomposed more clearly into a volatility and correlation component by adding and
subtracting ${{{\bfvarepsilon '}_t}\bfD_t^{ - 1}\bfD_t^{ - 1}{\bfvarepsilon _t}}={{{\bfz'}_t}{\bfz_t}}$,
\begin{equation}
\begin{gathered}
  LL = \frac{1}
{2}\sum\limits_{i = 1}^T {\left( {N\log \left( {2\pi } \right) + 2\log \left| {{\bfD_t}} \right| + \log \left| {{\bfR_t}} \right| + {{\bfz'}_t}\bfR_t^{ - 1}{{\bfz'}_t}} \right)}  \hfill \\
   = \frac{1}
{2}\sum\limits_{i = 1}^T {\left( {N\log \left( {2\pi } \right) + 2\log \left| {{\bfD_t}} \right| + {{\bfvarepsilon '}_t}\bfD_t^{ - 1}\bfD_t^{ - 1}{\bfvarepsilon _t}} \right) - } \frac{1}
{2}\sum\limits_{i = 1}^T {\left( {{{\bfz'}_t}{\bfz_t} + \log \left| {{\bfR_t}} \right| + {{\bfz'}_t}\bfR_t^{ - 1}{{\bfz'}_t}} \right)}  \hfill \\
   = {LL_V}\left( {{\theta _1}} \right) + {LL_R}\left( {{\theta _1},{\theta _2}} \right) \hfill \\
\end{gathered}
\end{equation}
where ${LL_V}\left( {{\theta _1}} \right)$ is the volatility component with
parameters $\theta_1$, and ${LL_R}\left( {{\theta _1},{\theta _2}} \right)$ the
correlation component with parameters $\theta_1$ and $\theta_2$. In the Multivariate
Normal case, where no shape or skew parameters enter the density, the volatility
component is the sum of the individual GARCH likelihoods which can be jointly
maximized by separately maximizing each univariate model. In other distributions,
such as the multivariate Student, the existence of a shape parameter means that
the estimation must be performed in one step so that the shape parameter is
jointly estimated for all models. Separation of the likelihood into 2 parts
provides for feasible large scale estimation. Together with the use of variance
targeting, very large scale systems may be estimated in a matter of seconds with
the use of parallel and grid computing. Yet as the system becomes larger and
larger, it becomes questionable whether the scalar parameters can adequately
capture the dynamics of the underlying process. As such, \cite{Cappiello2006}
generalize the DCC model with the introduction of the Asymmetric Generalized
DCC (\emph{AGDCC}) where the dynamics of $\bfQ_t$ are:
\begin{equation}\label{eq:agdcc1}
{\bfQ_t} = \left( {\bar \bfQ - \bfA'\bar Q\bfA - \bfB'\bar \bfQ\bfB - \bfG'{{\bar \bfQ}^ - }\bfG} \right) + \bfA'{\bfz_{t - 1}}{{\bfz'}_{t - 1}}\bfA + \bfB'{\bfQ_{t - 1}}\bfB + \bfG'\bfz_t^ - {{\bfz'}_t}^ - \bfG
\end{equation}
where $\bfA$, $\bfB$ and $\bfG$ are the $N\times N$ parameter matrices,
$\bfz_t^ -$ are the zero-threshold standardized errors which are equal
to $\bfz_t$ when less than zero else zero otherwise, $\bar \bfQ$ and $\bar \bfQ^ -$
the unconditional matrices of $\bfz_t$ and $\bfz_t^-$ respectively. Because of
its high dimensionality, restricted models have been used including the scalar,
diagonal and symmetric versions with the specifications nested being
\begin{itemize}
\item  DCC : $\bfG = \left[ 0 \right],\bfA = \sqrt a ,\bfB = \sqrt b$
\item ADCC : $\bfG = \sqrt g ,\bfA = \sqrt a ,\bfB = \sqrt b$
\item GDCC : $\bfG = \left[ 0 \right]$.
\end{itemize}
Variance targeting in such high dimensional models where the parameters are no
longer scalars, creates difficulties in imposing positive definiteness during
estimation while at the same time guaranteeing a global optimum solution. Methods
which directly check and penalize the eigenvalues of the intercept matrix introduce
non-smoothness and discontinuities into the likelihood surface for which inference
is likely to be difficult.\footnote{This has not prevented a plethora of paper
using these models and making inference based on questionable convergence
criteria.} More substantially, \cite{Aielli2009} points out that the
estimation of $\bar \bfQ_t$ as the empirical counterpart of the correlation
matrix of $\bfz_t$ in the DCC model is inconsistent since $\E[\bfz_t \bfz_t]=\E[\bfR_t]\neq \bfR[\bar \bfQ_t]$.
He proposes instead the cDCC model which includes a corrective step which eliminates
this inconsistency, albeit at the cost of targeting which is not allowed.\\
One model which tries to balance dimensionality with more realistic dynamics is the Flexible DCC (FDCC)
model of \cite{Billio2006} which allows groups \footnote{Groups here is used liberally,
and meant to denote membership in a set with common dynamics.} of securities to have the same
dynamics. The model may parsimoniously be represented as:
\begin{equation}\label{eq:fdcc}
{\bfQ_t} = cc' + \sum\limits_{j = 1}^P {\left( {{I_g}{a_j}} \right){{\left( {{I_g}{a_j}} \right)}^\prime } \odot {\varepsilon _{t - j}}} {\varepsilon '_{t - j}} + \sum\limits_{j = 1}^Q {\left( {{I_g}{b_j}} \right){{\left( {{I_g}{b_j}} \right)}^\prime } \odot {\bfQ_{t - j}}}
\end{equation}
where $I_g$ is the $assets\times groups$ logical matrix of group exclusive membership. This is
a very flexible representation allowing a large range of representations, from a single group
driving all dynamics (like the DCC), to each asset having its own group (like the GDCC).
Unfortunately, without specialized restrictions correlation targeting is lost, but the model
still remains feasible for a not too large number of groups. In the {\bf rmgarch} package,
the intercept is estimated \emph{using} correlation targeting with the intercept set to 
$({\mathbf{11'}} - aa' - bb')\odot \bar \bfQ$ and the restriction that $a_i a_j+b_i b_j<1,\forall i,j$ 
in order to avoid explosive patterns. Positive definiteness of the matrices is achieved by 
construction subject to a suitable starting point for $\bfQ_t$. Also note that only the 
FDCC(1,1) model is allowed (i.e. P=1, Q=1) because of the large number of pairwise constraints
needed which make higher order models prohibitively expensive to calculate (and in any case it is
quite rare to use anything beyond this for DCC type models).\\
In the {\bf rmgarch} package, the DCC, aDCC and FDCC models are implemented using the
2-stage approach, with a choice of 3 distributions, the multivariate Normal (\emph{MVN}),
Student (\emph{MVT}) and Laplace (\emph{MVL}). For the MVT distribution, it is
understood that this is based on known shape parameter (which may be fixed for
the first and second stage estimation using the fixed.pars method on the
specification object), else that the first stage estimation is QML based as in
\cite{Bauwens2005}.\\
\subsubsection{Forecasting}
Because of the nonlinearity of the DCC evolution process, the multi-step ahead
forecast of the correlation cannot be directly solved, and is instead based on the
approximation suggested in \cite{Engle2001}. Consider the multi-step ahead
evolution of the proxy process $Q_{t+n}$:
\begin{equation}
{Q_{t + n}} = \left( {1 - \alpha  - \beta } \right)\bar Q + \alpha {E_t}\left[ {{z_{t + n - 1}}{{z'}_{t + n - 1}}} \right] + \beta {Q_{t + n - 1}}
\end{equation}
where ${E_t}\left[ {{z_{t + n - 1}}{{z'}_{t + n - 1}}} \right] = {R_{t + n - 1}}$ and
${R_{t + n}} = diag{\left( {{Q_{t + n}}} \right)^{ - 1/2}}{Q_{t + n}}diag{\left( {{Q_{t + n}}} \right)^{ - 1/2}}$.
\cite{Engle2001} suggest 2 types of approximations possible to solve for $R_{t+n}$,
and the package adopts the one which, based on their findings, provides for the
least bias. That is, set $\bar Q \approx R$ and ${E_t}\left[ {{Q_{t + 1}}} \right] = {E_t}\left[ {{R_{t + 1}}} \right]$,
so that:
\begin{equation}
{E_t}\left[ {{R_{t + n}}} \right] = \sum\limits_{i = 0}^{n - 2} {\left( {1 - \alpha  - \beta } \right)\bar R{{\left( {\alpha  + \beta } \right)}^i} + {{\left( {\alpha  + \beta } \right)}^{n - 1}}{R_{n + 1}}}
\end{equation}
Importantly, for the rolling 1-ahead method, the estimate of $\bar Q$ at time T+n
is updated from data up to time T+n-1, which will lead to small differences in
results from applying the DCC filter method on new data (the difference will grow with n)
which uses a fixed value for $\bar Q$ (and which can be controlled by the n.old option).
\subsection{The GARCH-Copula Model}\label{section:cdcc}
Copula functions were introduced by \cite{Sklar1959} as a tool to connect disparate
marginal distribution together to form a joint multivariate distribution. They
were extensively used in survival analysis and the actuarial sciences for many
years before being introduced in the finance literature more than a decade ago
by \cite{Frey2000} and \cite{Li2000}. They have since been very popular in
investigating the dependence of financial time series of various assets classes
and frequencies. \cite{Breymann2003} investigate bivariate hourly FX spot returns
finding that the Student Copula best fit the data at all horizons (with the shape
parameter increasing with the time horizon), while \cite{Malevergne2003} find
that the Normal Copula fits pairs of currencies and equities well on the whole
but unsurprisingly fails to capture tail events where the Student Copula does
best.\footnote{Interestingly the authors argue that since such events are rare,
the goodness of fit test they use cannot always reject the Normal Copula.}
\cite{Junker2005} use a Frank copula with a transformation generator and GARCH
dynamics for the margins using the empirical distribution, to analyze the
bivariate dependency of the daily returns of 6 stocks and 3 Euro swap rates
(with horizons 2,5, and 10 Years). The comparison with a range of popular
copulas including the Normal and Student, in a risk exercise shows that
asymmetric tail dependency is important and usually not accommodated by the
Student distribution\footnote{An alternative would be to use the skew Generalized
Hyperbolic Student distribution analyzed in \cite{Aas2006} which allows
for the modelling of one heavy (with polynomial behavior) and one semi-heavy
(with exponential behavior) tail.} While most studies are predominantly focused
on bivariate copulas, the extension to n-variate models is not overtly
challenging particularly for elliptical distributions, or the use of the more
recent Vine pair copulas (see for example \cite{Joe2010}).
\subsubsection{Copulas}\label{copulas}
An n-dimensional copula $C\left( {{u_1},\dots,{u_n}} \right)$ is an n-dimensional
distribution in the unit hypercube ${\left[ {0,1} \right]^n}$ with uniform margins.
\cite{Sklar1959} showed that every joint distribution $F$ of the random vector
$\bfX=(x_1,\dots,x_n)$ with margins ${{F_1}\left( {{x_1}} \right),\dots,{F_n}\left( {{x_n}} \right)}$,
can be represented as:
\begin{equation}\label{eq:copula1}
F\left( {{x_1},\dots,{x_n}} \right) = C\left( {{F_1}\left( {{x_1}} \right),\dots,{F_n}\left( {{x_n}} \right)} \right)
\end{equation}
for a copula $C$, which is uniquely determined in ${\left[ {0,1} \right]^n}$
for distributions $F$ under absolutely continuous margins and obtained as:
\begin{equation}\label{eq:copula2}
C\left( {{u_1},\dots,{u_n}} \right) = F\left( {F_1^{ - 1}\left( {{u_1}} \right),\dots,F_n^{ - 1}\left( {{u_n}} \right)} \right)
\end{equation}
The density function may conversely be obtained as :
\begin{equation}\label{eq:copula3}
f\left( {{x_1},\dots,{x_n}} \right) = c\left( {{F_1}\left( {{x_1}} \right),\dots,{F_n}\left( {{x_n}} \right)} \right)\prod\limits_{i = 1}^n {{f_i}\left( {{x_i}} \right)}
\end{equation}
where $f_i$ are the marginal densities and $c$ is the density function of the
copula given by:
\begin{equation}\label{eq:copula4}
c\left( {{u_1},\dots,{u_n}} \right) = \frac{{f\left( {F_1^{ - 1}\left( {{u_1}} \right),\dots,F_n^{ - 1}\left( {{u_n}} \right)} \right)}}
{{\prod\limits_{i = 1}^n {{f_i}\left( {F_i^{ - 1}\left( {{u_i}} \right)} \right)} }}.
\end{equation}
with $F_i^{ - 1}$ being the quantile function of the margins. A key property of
copulas is their invariance under strictly increasing transformations of the
components of the $\bfX$, so that for example the copula of the multivariate
Normal distribution $F_n\left( {\bfmu ,\mbSigma } \right)$ is the same as
that of $F_n\left( {0,\bfR} \right)$ where $\bfR$ is the correlation matrix
implied by the covariance matrix, and the same for the copula of the multivariate
Student distribution reviewed in detail in \cite{Demarta2005}. The density
of the Normal Copula, of the n-dimensional random vector $\bfX$ in terms of the
correlation matrix $\bfR$, is then:
\begin{equation}\label{eq:copula5}
c\left( {u;\bfR} \right) = \frac{1}{{{{\left| \bfR \right|}^{1/2}}}}{e^{ - \frac{1}{2}u'\left( {\bfR' - \bfI} \right)u}}
\end{equation}
where ${\bfu_i} = {\mbPhi ^{ - 1}}\left( {{F_i}\left( {{bfx_i}} \right)} \right)$
for $i=1,\dots,n$, representing the quantile of the Probability Integral
Transformed (\emph{PIT}) values of $\bfX$, and $\bfI$ is the identity matrix.
Because the Normal Copula cannot account for tail dependence, the Student Copula
has been more widely used for modelling of financial assets. The density of the
Student Copula, of the n-dimensional random vector $\bfX$ in terms of the
correlation matrix $\bfR$ and shape parameter $\nu$, can be written as:
\begin{equation}\label{eq:copula6}
c\left( {\bfu;\bfR,\nu } \right) = \frac{{\mbGamma \left( {\frac{{\nu  + n}}{2}} \right){{\left( {\mbGamma \left( {\frac{\nu }
{2}} \right)} \right)}^n}{{\left( {1 + {\nu ^{ - 1}}\bfu'{\bfR^{ - 1}}\bfu} \right)}^{ - \left( {\nu  + n} \right)/2}}}}
{{{{\left| \bfR \right|}^{1/2}}{{\left( {\mbGamma \left( {\frac{{\nu  + n}}{2}} \right)} \right)}^n}\mbGamma \left( {\frac{\nu }
{2}} \right)\prod\limits_{i = 1}^n {{{\left( {1 + \frac{{\bfu_i^2}}{\nu }} \right)}^{ - \left( {\nu  + 1} \right)/2}}} }}
\end{equation}\label{eq:copula7}
where ${\bfu_i} = {t_\nu }^{ - 1}\left( {F\left( {{x_i};\nu } \right)} \right)$,
where $t_\nu^{-1}$ is the quantile function of the student distribution with
shape parameter $\nu$.
\subsubsection{Correlation and Kendall's $\bftau$}\label{II:kendall}
Pearson's product moment correlation $\bfR$ totally characterizes the dependence
structure in the multivariate Normal case, where zero correlation also implies
independence, but can only characterize the ellipses of equal density when the
distribution belongs to the elliptical class. In the latter case for instance,
with a distribution such as the multivariate Student, the correlation cannot
capture tail dependence determined by the shape parameter. Furthermore, it is
not invariant under monotone transformations of original variables making it
inadequate in many cases. An alternative measure which does not suffer from this
is Kendall's $\bftau$ (see \cite{Kruskal1958}) based on rank correlations
which makes no assumption about the marginal distributions but depends only on
the copula $C$. It is a pairwise measure of concordance calculated as:
\begin{equation}\label{eq:copula8}
\bftau \left( {{x_i},{x_j}} \right) = 4\int_0^1 {\int_0^1 {C\left( {{u_i},{u_j}} \right)dC\left( {{u_i},{u_j}} \right) - 1} }.
\end{equation}\label{eq:copula9}
For elliptical distributions, \cite{Lindskog2003} proved that there is a
one-to-one relationship between this measure and Pearson's correlation coefficient
$\rho$ given by:
\begin{equation}\label{eq:copula10}
\bftau \left( {{x_i},{x_j}} \right) = \left( {1 - \sum\limits_{x \in \mathbb{R}} {\left( {\mathbb{P}{{\left\{ {{X_i} = x} \right\}}^2}} \right)} } \right)\frac{2}
{\pi }\arcsin {\rho _{ij}}
\end{equation}
which under certain assumptions (such as in the case of the multivariate Normal)
simplifies to $\frac{2}{\pi}\arcsin {\rho_{ij}}$.\footnote{Another popular
measure is Spearman's correlation coefficient $\rho_s$ which under Normality
equates to $\frac{6}{\pi }\arcsin \frac{{{\rho _{ij}}}}{2}$, and it is usually
very close in result to Kendall's meaure.} Kendall's $\bftau$ is also invariant
under monotone transformations making it rather more suitable when working with
non-elliptical distributions. A useful application arises in the case of the
multivariate Student Distribution, where a maximum likelihood approach for the
estimation of the Correlation matrix $\bfR$ becomes unfeasible for large dimensions.
In this case, an alternative approach is to estimate the sample counterpart of
Kendall's $\bftau$\footnote{The matrix is build up from the pairwise estimates.}
from the transformed margins and then translate that into the correlation matrix
as detailed in \eqref{eq:copula10}, providing for a method of moments
type estimator.\footnote{It may be the case that the resultant matrix is not
positive definite, in which case a variety of methods exist to tweak it into one.}
The shape parameter $\nu$ may then be estimated keeping the correlation matrix
constant, with little loss in efficiency vis-a-vis the full maximum likelihood
method.\footnote{ According to at least one study of \cite{Zeevi2002}.}\\

\subsubsection{Transformations and Consistency}\label{II:transformation}
The estimation and PIT transformation of the margins provides for a great deal of
flexibility, with the possibility of adopting a parametric, semi-parametric or
empirical approach. The first method, whereby the margins and transformation are
performed using a parametric density, was termed the Inference-Functions-for-Margins
(\emph{IFM}) by \cite{Joe1997} who also established the asymptotic theory
for it. The semi-parametric method (\emph{SPD}) uses a distribution which couples
together generalized Pareto distribution (\emph{GPD}) fitted tails\footnote{For
which a Probability Weighted Moment approach exists which is quite robust.} with
a kernel based interior and described in \cite{Davison1990}, and offers a rather
flexible method for capturing fat tails observed in practise. Finally, the empirical
approach, also called pseudo-likelihood, was investigated by \cite{Genest1995}
and asymptotic properties established under the assumption that the sequence
of $\bfX$ is i.i.d.( see \cite{Durrleman2000} for an excellent summary of the
different methods and their properties.)\\
In the {\bf rmgarch} package, all 3 choices of transformations are available with
the SPD method using the {\bf spd} package of \cite{Ghalanos2012c}.

\subsubsection{The Student Copula AGDCC}\label{II:agdcccopula}
The extension of the static copula approach to dynamic models, and in particular
GARCH, was investigated by \cite{Patton2006} who extended and proved the
validity of Sklar\'s theorem for the conditional case. \cite{Jondeau2006}
combine the ACD model of \cite{Hansen1994} with skewed Student distribution
to model time-varying or regime switching Student Copula for the dependence
between pairs of countries, while \cite{Chollete2009} use a GARCH with
skewed Student distribution in the first stage and a regime switching model
with a Canonical vine copula for the high dependence regime and a Normal copula
for the low dependence regime. The use of the skewed Student distribution in
such models, beyond its tractability and desirable features, according to
\cite{Chollete2009} is so as to ensure that the asymmetry in the
dependence structure is purely the result of multivariate asymmetry and not an
artifact of poor modelling of the margins. \cite{Demarta2005} describe a
skewed Student copula derived from the Normal Mean Variance Mixture distribution
(described in the next section), with margins univariate skewed student
distributions with common shape ($\nu$) but separate skewness ($\gamma$)
parameters.\footnote{The reason for the common shape parameter is that the
mixing variable $\bfW$ in the Normal Mean Variance mixture is Inverse Gaussian
distribution, $W\sim Ig\left( {\nu /2,\nu /2} \right)$. A grouped type copula
whereby the shape parameter is also allowed to vary is also described by
\cite{Demarta2005}, in which case each variable has a different value for
the mixing variable $\bfW$, so that $\bfW_j\sim Ig\left( {\nu_j /2,\nu_j /2} \right)$,
for $j=1,\dots,n$, and the $\bfW_j$ are now perfectly correlated.}\\
In an elliptical distribution setting, adding dynamics to the correlation matrix
of the copula seems a natural extension of the 2-stage DCC model, and allows the
estimation of a Student copula with disparate shape parameters for the first stage,
where this was not possible using the standard DCC model (unless estimated jointly).
Let the n-dimensional random vector of asset returns $\bfr_t={r_{it},\dots,r_{nt}}$
follow a copula GARCH model with joint distribution given by:
\begin{equation}\label{eq:copula11}
F\left( {{\bfr_t}|{\bfmu _t},{\bfh_t}} \right) = C\left( {{F_1}\left( {{r_{1t}}|{\mu _{1t}},{h_{1t}}} \right),\dots,{F_n}\left( {{r_{nt}}|{\mu_{nt}},{h_{nt}}} \right)} \right)
\end{equation}
where $F_i$, $i=1,\dots,n$ is the conditional distribution of the $i^{th}$
marginal series density, $C$ is the n-dimensional Copula. The conditional mean
$\E\left[ {{r_{it}}\left| {{\Im _{t - 1}}} \right.} \right] = {\mu _{it}}$,
where $\Im _{t - 1}$ is the $\sigma$-field generated by the past realization of
$r_t$, and the conditional variance $h_{it}$ follows a GARCH(1,1)
process\footnote{For simplicity of exposition, a simple GARCH model is chosen,
but in fact any combination of GARCH models may be used.}:
\begin{align}\label{eq:copula12}
{r_{it}} &= {\mu _{it}} + {\varepsilon _{it}},{\varepsilon _{it}} = \sqrt {{h_{it}}} {z_{it}},\\
{h_{it}} &= \omega  + {\alpha _1}\varepsilon _{t - 1}^2 + \beta {h_{it - 1}}
\end{align}
where $z_{it}$ are i.i.d. random variables which conditionally follow a standardized
skew Student distribution, $z_{it}\sim {f}_{i}(0,1,\xi_i,\nu_i)$, of \cite{Fernandez1998}
with skew and shape parameters $\xi$ and $\nu$ respectively and derived in
the  {\bf {rugarch}} vignette.\footnote{Any combination of conditional distributions
can be used in the first stage. The skew-student is used here for illustration.}
The dependence structure of the margins is then assumed to follow a Student copula
with conditional correlation $\bfR_t$ and constant shape parameter $\eta$. The
conditional density at time $t$ is given by:
\begin{equation}\label{eq:copula13}
{c_t}\left( {{u_{it}},\dots,{u_{nt}}\left| {{\bfR_t},\eta } \right.} \right) = \frac{{{f_t}\left( {F_i^{ - 1}\left( {{u_{it}}\left| \eta  \right.} \right),\dots,F_i^{ - 1}\left( {{u_{nt}}\left| \eta  \right.} \right)\left| {{\bfR_t},\eta } \right.} \right)}}{{\prod\limits_{i = 1}^n {{f_i}\left( {F_i^{ - 1}\left( {{u_{it}}\left| \eta  \right.} \right)\left| \eta  \right.} \right)} }}
\end{equation}
where $u_{it}=F_{it}\left(r_it\left|\mu_{it},h_{it},\xi_{i},\nu_{i}\right.\right)$
is the PIT transformation of each series by its conditional distribution $F_{it}$
estimated via the first stage GARCH process, ${F_i^{ - 1}\left( {{u_{it}}\left| \eta  \right.} \right)}$
represents the quantile transformation of the uniform margins subject to the common
shape parameter of the multivariate density, ${{f_t}\left( .\left| {{\bfR_t},\eta } \right. \right)}$
is the multivariate density of the Student distribution with conditional correlation
$R_t$ and shape parameter $\eta$ and ${f_i}\left( .\left| \eta  \right.\right)$
is the univariate margins of the multivariate Student distribution with common
shape parameter $\eta$. The dynamics of $R_t$ are assumed to follow an AGDCC
model as described in the previous section, though it is more common to use a
restricted scalar DCC model for not too large a number of series. Finally, the
joint density of the 2-stage estimation is written as:
\begin{equation}\label{eq:copula14}
f\left( {{\bfr_t}\left| {{\bfmu _t},{\bfh_t}},{\bfR_t},\eta \right.} \right) = {c_t}\left( {{u_{it}},\dots,{u_{nt}}\left| {{\bfR_t},\eta } \right.} \right)\prod\limits_{i = 1}^n {\frac{1}
{{\sqrt {{h_{it}}} }}{f_{it}}\left( {{z_{it}}\left| {{\nu _i},{\xi _i}} \right.} \right)}
\end{equation}
where it is clear that the likelihood is composed of a part due to the joint DCC
copula dynamics and a part due to the first stage univariate GARCH dynamics.\\
A similar model, with Student margins, was estimated by \cite{Ausin2010}
using a Bayesian setup, and an empirical risk management application, albeit once
again using only a bivariate series (DAX and Dow Jones indices), used to illustrate
its applicability and appropriateness.\\
In the {\bf rmgarch} package, the Normal and Student copulas are implemented,
with either a static or dynamic correlation model (aDCC).
\subsubsection{Forecasting}
Because of the nonlinear transformation of the margins, there is no closed
form solution for the multi-step ahead forecast. As such, the \emph{cgarchsim}
method must be used. The inst folder of the package contains a number of examples.

\subsection{The GO-GARCH Model}\label{section:gogarch}
Factor ARCH models, originally introduced by \cite{Engle1990} and with
foundations in the Arbitrage Pricing Theory of \cite{Ross1976}, are based
on the assumption that returns are generated by a set of unobserved underlying
factors that are conditionally heteroscedastic. The dependence framework is
non-dynamic as a consequence of large scale estimation in a multivariate setting.
The dependence structure of the unobserved factors then determines the type of
factor model it belongs to, with correlated factors making up the F-ARCH type
models, while uncorrelated and independent factors comprise the Orthogonal and
Generalized Orthogonal Models respectively.\footnote{It should be noted, that
most of these factor models may be seen as special cases of the BEKK model.
The GO-GARCH model has the following restricted BEKK representation:
\begin{equation}
{H_t} = C + \sum\limits_{i = 1}^m {{A_i}{x_{t - 1}}{{x'}_{t - 1}}} {{A'}_i} + B{H_{t - 1}}B'.
\end{equation}
Under the assumption that all $A_i$ and B are restricted to have the same
eigenvector $Z$, with the eigenvalues of A being all zero except the $i^{th}$
one, and the C can be decomposed into $ZDZ'$ where $D$ is some positive definite
diagonal matrix, then this is a GO-GARCH (with GARCH(1,1) univariate dynamics)
model where Z is the linear ICA map. However, GO-GARCH model is not limited to
GARCH(1,1) or any particular process for the factors.} Because one can always
re-discover uncorrelated or independent sources by certain statistical transformation,
the correlated factor assumption of F-ARCH models does appear to be restrictive.
GO-GARCH models on the other hand make use of those transformations to place the
factors in an independence framework with unique benefits such as separability
and weighted density convolution giving rise to truly large scale, real-time and
feasible estimation. Consider a set of $N$ assets whose returns \( \bfr_t \) are
observed for $T$ periods, with  conditional mean \(E[\bfr_t|\mathfrak{F}_{t-1}] = \bfm_t\),
where \(\mathfrak{F}_{t-1} \) is the \(\sigma\)-field generated by the past
realizations of \( \bfr_t \), i.e. \( \mathfrak{F}_{t-1} = \sigma(\bfr_{t-1},\bfr_{t-2},\ldots) \).
The GO-GARCH Model of \cite{Weide2002} maps \( \bfr_t  -\bfm_t\) onto a
set of unobserved independent factors \( \bff_t \) (or "structural errors"),
\begin{eqnarray}
\bfr_t & = & \bfm_t + \bfepsilon_t \quad t = 1,\ldots, T \\
\bfepsilon_t & = & \bfA \bff_t,
\end{eqnarray}
where \( \bfA \) is invertible and constant over time and may be decomposed into
the de-whitening matrix $\mbSigma^{1/2}$, representing the square root of the
unconditional covariance, and an orthogonal matrix, $\bfU$, so that:
\begin{equation}\label{eq:matrix_A}
\bfA = \mathbf{\Sigma}^{1/2}\bfU,
\end{equation}
and \(\bff_t = (f_{1t},\ldots,f_{Nt})'\). The rows of the mixing matrix \(\bfA \)
therefore represent the independent source factor weights assigned to each asset
(i.e. rows are the assets and columns the factors). The factors have the following
specification:
\begin{equation}
    \bff_{t} = \bfH_t^{1/2} \bfz_t,
\end{equation}
where \( \bfH_t = E[\bff_t \bff_t' | \mathfrak{F}_{t-1}] \) is a diagonal matrix
with elements \((h_{1t},\ldots,h_{Nt} )\) which are the conditional variances of
the factors, and \(\bfz_t = (z_{1t},\ldots,z_{Nt})' \). The random variable \( z_{it} \)
is independent of \( z_{jt-s} \) \(\forall j \neq i \) and \( \forall s\),
with \(E[z_{it}|\mathfrak{F}_{t-1}]=0\) and \( E[z_{it}^2]=1 \), this implies
that \( E[\bff_t|\mathfrak{F}_{t-1}] = \mathbf{0} \)
and \(E[\bfepsilon_t|\mathfrak{F}_{t-1}] = \mathbf{0}\). The factor conditional
variances, \(h_{i,t}\), can be modelled as a GARCH-type process. The unconditional
distribution of the factors is characterized by:
\begin{equation}\label{eq:factor_moments}
E[\bff_t] = \mathbf{0} \quad  E[\bff_t\bff_t'] = \bfI_N
\end{equation}
which, in turn, implies that:
\begin{equation}
E[\bfepsilon_t] = \mathbf{0} \quad E[\bfepsilon_t\bfepsilon_t'] = \bfA \bfA'.
\end{equation}
It follows that the returns can be expressed as:
\begin{equation}\label{eq:returns_model}
\bfr_t = \bfm_t + \bfA \bfH_t^{1/2} \bfz_t.
\end{equation}
The conditional covariance matrix, \( \mbSigma_t \equiv E[(\bfr_t -\bfm_t)(\bfr_t - \bfm_t)'|\mathfrak{F}_{t-1}] \)
of the returns is given by:
\begin{equation}\label{eq:ifacdcov}
    \mbSigma_t  =  \bfA \bfH_t \bfA'.
\end{equation}
The Orthogonal Factor model of \cite{Alexander2001}\footnote{When $\bfU$
is restricted to be an identity matrix, the model reduces to the Orthogonal
Factor model.} which uses only information in the covariance matrix, leads to
uncorrelated components but not necessarily independent unless assuming a
multivariate normal distribution. However, while whitening is not sufficient
for independence, it is nevertheless an important step in the preprocessing of
the data in the search for independent factors, since by exhausting the second
order information contained in the covariance matrix it makes it easier to infer
higher order information, reducing the problem to one of rotation (orthogonalization).
The original procedure of \cite{Weide2002} used a 1-step maximum likelihood
approach to jointly estimate the rotation matrix and dynamics making the procedure
infeasible for anything other than a few assets. Alternative approaches such as
nonlinear least squares and method of moments for the estimation of $\bfU$ have
been proposed in \cite{Weide2004} and \cite{Weide2008}, respectively.
In the {\bf rmgarch} package, I estimate the matrix \( \bfU \) by ICA as in
\cite{Broda2009} and \cite{Zhang2009}. One of the computational
advantages offered by the Generalized Orthogonal approach is that following the
estimation of the independent factors, the dynamics of the marginal density
parameters of those factors may be estimated separately.
\subsubsection{ICA}
The estimation of the factor loading matrix \( \bfA \) exploits the decomposition
in \eqref{eq:matrix_A}. The estimation of ${\mathbf{\Sigma}^{1/2}}$,
representing the square root of the unconditional covariance matrix, is usually
obtained from the OLS residuals \( \widehat{\bfepsilon}_t = \bfr_t - \widehat{\bfm}_t \),
while the orthogonal matrix $\bfU$ can be estimated using ICA (see \cite{Broda2009}, \cite{Zhang2009}).
ICA is a computational method for separating multivariate mixed signals, $\bfx=[x_1, ..., x_{n}]'$,
into additive statistically independent and non-Gaussian components, $\bfs=[s_1, ..., s_{n}]'$,
such that $\bfx=\bfB \bfs$. The objective is to decompose the observed $\bfx=[x_1, ..., x_{n}]'$,
into independent factors $\bfs=[s_1, ..., s_{n}]'$ and a linear matrix $\bfB$,
such that $\bfx=\bfB \bfs$. The independent source vector $\bfs \in {\mathbb{R}^n}$,
is assumed to be sampled from a joint distribution $f(\bfs)$,
\begin{equation}\label{eq:ica1}
f({s_1},...,{s_n}) = f({s_1})f({s_2})...f({s_n}),
\end{equation}
where $\bfs$ is not directly observable, nor is the particular form of the individual
distributions, $f(s_i)$, usually known.\footnote{If the distributions are known the
problem reduces to a classical maximum likelihood parametric estimation.} This forms
the key property of independence, namely that the joint density of independent signals
is simply the product of their margins. The estimate of the linear mixing matrix
$\bfB$ can be obtained via estimation methods based on a choice of criteria for
measuring independence which include the maximization of non-Gaussianity through
measures such as kurtosis and negentropy, minimization of mutual information,
likelihood and infomax. This follows from the Central Limit Theorem which states
that mixtures of independent variables tend to become more Gaussian in distribution
when they are mixed linearly, hence maximizing non-Gaussianity leads to independent
components (see \cite{Hyvaerinen2000} for more details).\footnote{Estimation by
minimization of the mutual information was first proposed by \cite{Comon1994}
who derived a fundamental connection between cumulants, negentropy and mutual information.
The approximation of negentropy by cumulants was originally considered much earlier
in \cite{Jones1987}, while the connection between infomax and likelihood
was shown in \cite{Pearlmutter1997}, and the connection between mutual
information and likelihood was explicitly discussed in \cite{Cardoso2000}}
Entropy may be thought of as the amount of information inherent within a random
variable, being an increasing function of the amount of randomness in that variable.
For a discrete random variable $X$ it is defined as,
\begin{equation}\label{eq:entropy1}
H(X) =  - \sum\limits_i {P(X = {b_i})\log P(X = {b_i}} ),
\end{equation}
with $b_i$ denoting the possible values of $X$. In the continuous case, for a continuous
random variable $X$ with density $f_X(x)$, the entropy\footnote{In the continuous case
this is usually called \textit{differential entropy}.} $H$ is defined as,
\begin{equation}\label{eq:entropy2}
H(X) =  - \int {f_{X}(x)\log f_{X}(x)\mathrm{d}x}.
\end{equation}
A key result from information theory states that among all random variables of
equal variance, a Gaussian variable has the largest entropy. Hence entropy could
be used as a measure of non-Gaussianity. A related measure of non-Gaussianity is
the negentropy which is always non-negative and zero for a Gaussian variable. It
is defined as,
\begin{equation}\label{eq:entropy3}
J(X) = H({X_{gauss}}) - H(X),
\end{equation}
where $H({X_{gauss}})$ is the entropy of a Gaussian random variable having the same
covariance matrix as $X$. As shown by \cite{Comon1994}, negentropy is invariant
for invertible linear transformations and is an optimal estimator of non-Gaussianity
with regards to its statistical properties (i.e. consistency, asymptotic variance
and robustness). In practice, because we do not know the density, approximations of
negentropy are used such as the one by
\cite{Hyvaerinen2000},
\begin{equation}\label{eq:entropy4}
J(X) \approx \sum\limits_{i = 1}^p {{k_i}{{[E({G_i}(X)) - E({G_i}(V))]}^2}},
\end{equation}
where $k_i$ are positive constants, $V$ is a standardized Gaussian variable and
$G_i$ are non-quadratic functions. The choice of the non-quadratic function has
an impact on the robustness of the estimators of negentropy. with $G(x)=x^4$
(kurtosis based) being the least robust while more robust measures would include,
\begin{equation}\label{eq:entropy5}
{g_1}(u) = \frac{1}{{{a_1}}}\log \cosh {a_1}u,\qquad {g_2}(u) =  - \exp ( - 0.5{u^2}).
\end{equation}
Because these non-quadratic functions present a complex nonlinear optimization problem,
sophisticated numerical algorithms are usually necessary. Two main algorithms are used,
the online and batch methods, with the former based on stochastic gradient methods
while in the latter case a popular choice is the natural gradient ascend of likelihood.
The FastICA of \cite{Hyvaerinen2000} is a very efficient batch algorithm with
a range of options for the non-quadratic functions. It can be used to estimate
the components either one at a time by finding maximally non-Gaussian directions
or in parallel by maximizing non-Gaussianity or the likelihood. The estimation
procedure of the GO-GARCH model can be summarized as follows. First, the FastICA
is applied to the whitened data \( \bfz_t = \widehat{\mbSigma}^{-1/2}\widehat{\bfepsilon}_t \),
where \( \widehat{\mbSigma}^{1/2} \) is obtained from the eigenvalue decomposition
of the OLS residual covariance matrix, returning an estimate of \( \bff_t\), i.e.,
\(\bfy_t = \bfW \bfz_t\). Second, because of the assumption of independence, the
likelihood function of the GO-GARCH model is greatly simplified so that the conditional
log-likelihood function is expressed as the sum of the individual conditional
log-likelihoods, derived from the conditional marginal densities of the factors,
i.e., $GH_{\lambda_i}(y_{it})\equiv GH(y_{it};\lambda_i,\mu_{i}\sqrt{h_{it}},\delta_{i}\sqrt{h_{it}},\alpha_{i}/\sqrt{h_{it}},\beta_{i}/\sqrt{h_{it}})$,
plus a term for the mixing matrix \(\bfA\), estimated in the first step by FastICA:
\begin{equation}\label{eq:log_lik_IFACD}
L(\widehat{\bfepsilon} _t \left| \bftheta ,\bfA \right.) = T \log{\left| \bfA^{ - 1} \right|} +
\sum\limits_{t = 1}^T \sum\limits_{i = 1}^N \log{\left(GH_{\lambda_i}(y_{it}|\theta_i)\right)}
\end{equation}
where \( \bftheta \) is the vector of unknown parameters in the marginal densities.
Because ICA is a linear noiseless model,\footnote{According to \cite{Hyvaerinen2000},
this can be partially justified by the fact that most of the research on ICA has also
concentrated on the noise free model and it has been shown with overwhelming empirical
support across a number of different disciplines to be a very good approximation to
a more complex model with noise added. Because the estimation of the noise-free model
has proved to be a very difficult task in itself, the noise-free model may also be
considered a tractable approximation of the more realistic noisy model.} the implication
for this 2 stage estimation in the GO-GARCH model is that uncertainty plays no part
in the derivation of the mixing matrix \(\bfA\) and hence does not affect the
standard errors of the independent factors.\\
The possibility of modelling the independent factors separately not only increases
the flexibility of the model but also its computational feasibility, since the
multivariate estimation reduces to \emph{N} univariate optimization steps plus a
term which depends on the factor loading matrix. Thus the independence property
of the model allows the estimation of very large scale systems on modern
computational grids with the time required to calculate any n-dimensional
model equivalent to the time it takes to estimate one single factor in this
framework.\\
In the {\bf rmgarch} package, 2 algorithms for ICA are implemented locally. The
FastICA of \cite{Hyvaerinen2000}, based on a direct translation of their Matlab
code and the RADICAL of \cite{Learned-Miller2003} which offers a robust
alternative. Both models allow a choice of common options such as the type of
covariance estimator to use for the whitening stage (e.g. Ledoit-Wolf, EWMA) as
well as the possibility of dimensionality reduction during the PCA stage. In the
latter case, some results for the model are still to be derived and it is therefore
considered experimental at this stage.
\subsubsection{Conditional Co-Moments}\label{II}
It seems to be a well-established, stylized fact that the unconditional security
return distribution is not normal and the mean and variance of returns alone are
insufficient to characterize the return distribution completely. This has led
researchers to pay attention to the third moment - skewness - and the  fourth
moment - kurtosis. The validity of the CAPM in the presence of higher-order
co-moments and their effects on asset prices has been thoroughly investigated.
The simple, single-factor, CAPM only holds under very specific conditions.
When asset prices are non-normal and investors have non-quadratic preferences,
then they will care about all return moments and not only mean and variance,
as in the standard CAPM.
There are a number of extensions to the basic two-moments CAPM which predict a
linear relationship in which terms like co-skewness and co-kurtosis are priced.
For example, \cite{Kraus1976}, \cite{Sears1985} extended the CAPM to
incorporate skewness in asset valuation models but provided mixed results. A few
studies have shown that non-diversified skewness and kurtosis play an important
role in determining security valuations. \cite{Fang1997}, derived a four-moment
CAPM where it was shown that systematic variance, systematic skewness and systematic
kurtosis contribute to the risk premium of an asset. \cite{Harvey2000}
examined an extended CAPM, including systematic co-skewness,  reporting that
conditional skewness explains the cross-sectional variation of expected returns
across assets and is significant even when factors based on size and book-to-market
are included. As skewness of a portfolio matters to investors, an asset's contribution
to the skewness of a broadly diversified portfolio, referred to as "co-skewness"
with the portfolio, may also be rewarded. Skewness preference further suggests that
the representative investor may adjust his diversified portfolio such that an individual
security's contribution to the skewness of the market portfolio may become a component
of the security's expected returns.
Mathematically, as demonstrated in \cite{Conine1981}, both individual assets'
skewness  and co-skewness between assets contribute to the skewness of the portfolio
which is composed of these assets. Intuitively, as positive (negative) skewness
implies a probability of obtaining a large positive (negative) return (relative
to a benchmark such as the normal distribution), a positive co-skewness of an asset
with another asset means that, when the price volatility goes up the return of
this asset also goes up. The general acceptance that the conditional density of
asset returns is not completely and adequately  characterized by the first two
moments, implies that the derivation of any measure of risk from that density
requires estimates for the higher order co-moments of the return distribution if
one is work within a multivariate setting. The linear affine representation of
the GO-GARCH model allows to identify closed-form expression for the conditional
co-skewness and co-kurtosis of asset returns\footnote{It is possible to go beyond
these moments but the notation becomes cumbersome and the benefits likely to
be marginal.}, as described in \cite{Athayde2000}. The conditional co-moments
of \( \bfr_t \) of order 3 and 4  are represented as tensor matrices,
\begin{equation}\label{eq:facd6}
\begin{gathered}
  \bfM_{t}^3 = \bfA \bfM_{f,t}^3(\bfA' \otimes \bfA'), \hfill \\
  \bfM_{t}^4 = \bfA \bfM_{f,t}^4(\bfA' \otimes \bfA' \otimes \bfA'), \hfill \\
\end{gathered}
\end{equation}
where \( \bfM_{f,t}^3\) and \( \bfM_{f,t}^4 \) are the \( (N \times N^2) \)
conditional third co-moment matrix and the \( (N \times N^3) \) conditional
fourth co-moment matrix of the factors, respectively. \( \bfM_{f,t}^3 \)
and \( \bfM_{f,t}^4\), defined as are given by
\begin{eqnarray}
\bfM_{f,t}^3 & =&
\begin{bmatrix}
\bfM_{1,f,t}^3,\bfM_{2,f,t}^3,\ldots,\bfM_{N,f,t}^3
\end{bmatrix} \label{eq:comoment_factor_matrix_3}\\
\bfM_{f,t}^4 & = &
\begin{bmatrix}
\bfM_{11,f,t}^4,\bfM_{12,f,t}^4,\ldots,\bfM_{1N,f,t}^4|\ldots|\bfM_{N1,f,t}^4,\bfM_{N2,f,t}^4,\ldots,\bfM_{NN,f,t}^4
\end{bmatrix}\label{eq:comoment_factor_matrix_4}
\end{eqnarray}
where \( \bfM_{k,f,t}^3, k=1,\ldots,N \) and \( \bfM_{kl,f,t}^4, k,l=1,\ldots,N \)
are the \((N\times N)\) submatrices of \( \bfM_{f,t}^3 \) and \( \bfM_{f,t}^4\),
respectively, with elements
\begin{eqnarray*}
m_{ijk,f,t}^3 & = & E[f_{i,t}f_{j,t}f_{k,t}|\mathfrak{F}_{t-1}] \\
m_{ijkl,f,t}^4 & = & E[f_{i,t}f_{j,t}f_{k,t}f_{l,t}|\mathfrak{F}_{t-1}].
\end{eqnarray*}
Since the factors $f_{it}$ can be decomposed as $z_{it}\sqrt{h_{it}}$, and given
the assumptions on \( z_{it} \), then \(E[f_{i,t}f_{j,t}f_{k,t}|\mathfrak{F}_{t-1}] = 0\).
It is also true that for \( i \neq j\neq k \neq l \)\(E[f_{i,t}f_{j,t}f_{k,t}f_{l,t}|\mathfrak{F}_{t-1}] = 0\)
and when \(i=j\) and \(k=l\),
\begin{equation*}
E[f_{i,t}f_{j,t}f_{k,t}f_{l,t}|\mathfrak{F}_{t-1}]  =  h_{it}^2 h_{kt}^2.
\end{equation*}
Thus, under the assumption of mutual independence, all elements in the conditional
co-moments matrices with at least 3 different indices are zero. Finally, standardizing
the conditional co-moments one obtains conditional co-skewness and co-kurtosis of $\bfr_t$,
\begin{equation}\label{eq:facd7}
\begin{gathered}
  \bfS_{ijk,t} = \frac{m_{ijk,t}^3}{({\sigma_{i,t}}{\sigma _{j,t}}{\sigma _{k,t}})}, \hfill \\
  \bfK_{ijkl,t} = \frac{m_{ijkl,t}^4}{({\sigma_{i,t}}{\sigma _{j,t}}{\sigma _{k,t}}{\sigma _{l,t}})}, \hfill \\
\end{gathered}
\end{equation}
where $ \bfS_{ijk,t}$ represents the asset co-skewness between elements
$i,j,k$ of $\bfr_t$, $\sigma_{i,t}$ the standard deviation  of $\bfr_{i,t}$, and
in the case of $i=j=k$ represents the skewness of asset \(i\) at time $t$, and
similarly for the co-kurtosis tensor $\bfK_{ijkl,t}$. Two natural applications
of return co-moments matrices are Taylor type utility expansions in portfolio
allocation and higher moment news impact surfaces. In the {\bf rmgarch} package
the covariance, correlation, coskewness and cokurtosis can be extracted from any
of the returned GO-GARCH objects (\emph{goGARCHfit}, \emph{goGARCHfilter},
\emph{goGARCHforecast}, \emph{goGARCHsim} \emph{goGARCHroll}) by using the methods
\emph{rcov}, \emph{rcor}, \emph{rcoskew} and \emph{rcokurt}, respectively.
Additional arguments to these methods are clearly detailed in the help files.
To obtain the weighted portfolio moments, using the geometric properties of the
model, the method \emph{gportmoments} can be called on any of the GO-GARCH
objects together with a weighting matrix.

\subsubsection{The Portfolio Conditional Density}\label{III}
An important question that can be addressed in this framework is the determination
of the portfolio conditional density, an issue of vital importance in risk
management application. The \emph{N}-dimensional NIG distribution, closed under
convolution, is suited to problems in portfolio and risk management where a
weighted sum of assets is considered. However, when the distributional parameters
$\alpha$ and $\beta$, representing skew and shape, are allowed to vary per asset,
as in the GO-GARCH case, this property no longer holds and numerical methods
such as that of the Fast Fourier Transform (\emph{FFT}) are needed to derive the
weighted density by inversion of the characteristic function of the scaled
parameters\footnote{This effectively means that the weighted density is not
necessarily NIG distributed.}. In the case of the NIG distribution, this is greatly
simplified because of the representation of the modified Bessel function for the GIG
shape index ($\lambda$) with value $-0.5$ which was derived in \cite{Barndorff-Nielsen1981},
otherwise the characteristic function of the GH involves the evaluation of the
modified Bessel function with complex arguments, which complicates the inversion.
Appendix \ref{appendixI} derives the characteristic functions used in the case
of independent margins for both the NIG and full GH distributions. Let $R_t$ be
the portfolio return:
\begin{align}\label{eq:portfolio1}
{R_t} & =  \bfw'_t \bfr_t = \bfw'_t \bfm_t + (\bfw_t' \bfA \bfH_t^{1/2}){\bfz_t}
\end{align}
where $\bfH _t^{1/2}$ is estimated from the GARCH dynamics of $\bfy_t$. The model
allows to express the portfolio variance, skewness and kurtosis in closed form,
\begin{equation}\label{eq:portfolio5}
\begin{gathered}
  \sigma_{{p,t}}^2 = \bfw_t'{\mbSigma_{t}}\bfw_t, \hfill \\
  {s_{p,t}} = \frac{{\bfw_t'\bfM_{_t}^3(\bfw_t \otimes \bfw_t)}}{{{{(\bfw_t'{\mbSigma _{t}}\bfw_t)}^{3/2}}}}, \hfill \\
  {k_{p,t}} = \frac{{\bfw_t'\bfM_{_t}^4(\bfw_t \otimes \bfw_t \otimes \bfw_t)}}{{{{(\bfw_t'{\mbSigma_{t}}\bfw_t)}^2}}}, \hfill \\
\end{gathered}
\end{equation}
where $\mbSigma_{t}$, $\bfM_{t}^3$ and $\bfM_{t}^4$ are derived in
\eqref{eq:facd6}. The portfolio conditional density may be obtained via
the inversion of the characteristic function through the FFT method as in
\cite{Chen2007} (see Appendix \ref{appendixI} for details) or by
simulation. The former is used in this package for its accuracy and speed.
Provided that $\bfz_t$ is a \emph{N}-dimensional vector of innovations,
marginally distributed as 1-dimensional standardized GH, the density of weighted
asset return, ${w_{it}}{r_{it}}$, is
\begin{equation}\label{eq:portfolio2}
{w_{i,t}}{r_{i,t}} = ({w_{i,t}}{m_{i,t}} + \overline{w}_{i,t}{z_{i,t}}) \sim
GH_{\lambda_i}\left(\overline{w}_{i,t}\mu_{i}+
{w_{i,t}}m_{i,t},\left|\overline{w}_{i,t}\right|\delta_{i},\frac{\alpha_{i}}{\left|
\overline{w}_{i,t} \right|},\frac{\beta_{i}}{\left| \overline{w}_{i,t} \right|}\right)
\end{equation}
where $\overline{\bfw}_t'$ is equal to $\bfw_t' \bfA \bfH_t^{1/2}$,
and \(\overline{w}_{i,t} \) is the \emph{i}-th element of \( \overline{\bfw}_t\), $m_{i,t}$
the conditional mean of the \emph{i}-th underlying asset. In order to obtain the
density of the portfolio, we must sum the individual weighted densities of
$z_{i,t}$.  The characteristic function of the portfolio return $R_t$ is
\begin{equation}\label{eq:portfolio3}
\varphi_R(u) = \prod\limits_{i = 1}^n {{\varphi_{\bar w{Z_i}}}} (u)
=  \exp{
   \left(
   \text{i}u\sum\limits_{j = 1}^d \bar\mu_j +
   \sum\limits_{j = 1}^d
   \left(
   \frac{\lambda_j}{2}
   \log{\left(\frac{\gamma}{\upsilon}\right)}
   +
   \log \left(
   \frac{\bfK_{\lambda _j}(\bar\delta_j\sqrt{\upsilon})}{\bfK_{\lambda_j}( \bar\delta_j \sqrt{\gamma})}
   \right)
   \right)
   \right)
   }
\end{equation}
where, $\gamma  = \bar \alpha _j^2 - \bar \beta _j^2$, $\upsilon  = \bar \alpha _j^2 - {({{\bar \beta }_j} + {\text{i}}u)^2}$,
and $(\bar \alpha_j, \bar \beta_j, \bar \delta_j, \bar \mu_j)$ are the scaled
versions of the parameters $(\alpha_{i}, \beta_{i}, \delta_{i}, \mu_{i})$
as shown in \eqref{eq:portfolio2}. The density may be accurately
approximated by FFT as follows,
\begin{equation}\label{eq:portfolio4}
{f_R}(r) = \frac{1} {{2\pi }}\int_{ - \infty }^{ + \infty } {{e^{( - \text{i}u r)}}} \varphi_R
(u)\mathrm{d}u \approx \frac{1} {{2\pi }}\int_{ - s}^s {{e^{( - \text{i}u r)}}} \varphi_R
(u)\mathrm{d}u.
\end{equation}
Once the density is formed by FFT inversion of the characteristic function,
distribution, quantile and sampling functions can be created. In the
{\bf rmgarch} package these are represented are \emph{dfft}, \emph{pfft},
\emph{qfft} and \emph{rfft}, which operate on the point in time conditional
density approximation, an object of class \emph{goGARCHfft}, returned from
calling the \emph{convolution} method on a fitted (\emph{goGARCHfit}), filtered
(\emph{goGARCHfilter}), forecasted (\emph{goGARCHforecast}), simulated
(\emph{goGARCHsim}) or rolling (\emph{goGARCHroll}) object. Finally, the
\emph{nportmoments} method applied to a \emph{goGARCHfft} object will return
the FFT-based semi analytic portfolio moments.
\subsubsection{Forecasting}
The multi-step ahead forecast of the GO-GARCH model is based completely on the
univariate factor dynamics, already covered in the {\bf {rugarch}} package.
Additionally, all methods available for working with a fitted (\emph{goGARCHfit})
object are also available for the resulting forecast (\emph{goGARCHforecast})
object and covered in detail in the help file, and the examples in the inst
folder of the package.

\section{Miscellaneous}
Like the {\bf rugarch} package, parallel functionality is implemented by passing a
pre-created cluster object from the parallel package. Unlike the {\bf rugarch}
package, there is a much higher cost to the use of a socket (snowfall) rather than
fork (multicore) based setup, and depending on the number of sockets used, it may be
the case that the data communication overhead is so high that non-parallel estimation
is faster.\\
A comprehensive set of examples is available in the rmgarch.tests folder of the source. 
There are 5 main files, covering the Copula, DCC, FDCC and GO-GARCH models and the fScenario 
and fMoments methods for use in portfolio and risk management applications (see the {\bf parma}
package).\\
\clearpage
\appendixpage
\addappheadtotoc
\subsection*{The GH characteristic function}\label{appendixI}
The moment generating function (\emph{MGF}) of the GH Distribution is,
\begin{equation}\label{appendixI:eq:ghypmom}
\begin{gathered}
  {M_{GH(\lambda ,\alpha ,\beta ,\delta ,\mu )}}(u) = {e^{\mu u}}{M_{GIG\left( {\lambda ,\delta \sqrt {{\alpha ^2} - {\beta ^2}} } \right)}}\left( {\frac{{{u^2}}}{2} + \beta u} \right), \hfill \\
   = {e^{\mu u}}{\left( {\frac{{{\alpha ^2} - {\beta ^2}}}{{{\alpha ^2} - {{(\beta  + u)}^2}}}} \right)^{\lambda /2}}\frac{{{\bfK_\lambda }\left( {\delta \sqrt {{\alpha ^2} - {{(\beta  + u)}^2}} } \right)}}{{{\bfK_\lambda }\left( {\delta \sqrt {{\alpha ^2} - {\beta ^2}} } \right)}} \hfill \\
\end{gathered}
\end{equation}
where $M_{GIG}$ represents the moment generating function of the Generalized Inverse
Gaussian which forms the mixing distribution in this variance-mean mixture subclass.
Powers of the MGF, $M_{GH}(u)^p$, only have the representation in \eqref{appendixI:eq:ghypmom}
for $p=1$, which means that GH distributions are not closed under convolution
with the exception of the NIG, and only in the case when the shape and skew parameters
are the same. The MGF of the NIG is,
\begin{equation}\label{appendixI:eq:nigmom}
{M_{NIG(\alpha ,\beta ,\delta ,\mu )}}(u) = {e^{\mu u}}\frac{{{e^{\delta \sqrt {{\alpha ^2} - {\beta ^2}} }}}}{{{e^{\delta \sqrt {{\alpha ^2} - {{(\beta  + u)}^2}} }}}}.
\end{equation}
Powers of $p$ are equivalent in this case to multiplication by $p$ of $\delta$ and $\mu$, so that,
\begin{equation}\label{appendixI:eq:closednig}
NIG(\alpha ,\beta ,{\delta _1},{\mu _1}) \times...\times NIG(\alpha ,\beta ,{\delta _n},{\mu _n}) = NIG(\alpha ,\beta ,{\delta _1} + ... + {\delta _n},{\mu _1} + ... + {\mu _n}).
\end{equation}
When the distribution is not closed under convolution, numerical methods are
required such as the inversion of the characteristic function by FFT. Because the
MGF is a holomorphic function for complex $z$, with $\left| z \right| < \alpha  - \beta$,
we can obtain the characteristic function of the GH distribution, using the following
representation,
\begin{equation}\label{appendixI:eq:chartrick}
{\phi _{GH}}(u) = {M_{GH}}({\text{i}}u),
\end{equation}
so that the characteristic function may be written as,
\begin{equation}\label{appendixI:eq:ghypchar}
{\phi _{GH(\lambda ,\alpha ,\beta ,\delta ,\mu )}}(u) = {e^{\mu \text{i}u}}{\left( {\frac{{{\alpha ^2} - {\beta ^2}}}{{{\alpha ^2} - {{(\beta  + {\text{i}}u)}^2}}}} \right)^{\lambda /2}}\frac{{{\bfK_\lambda }\left( {\delta \sqrt {{\alpha ^2} - {{(\beta  + {\text{i}}u)}^2}} } \right)}}{{{\bfK_\lambda }\left( {\delta \sqrt {{\alpha ^2} - {\beta ^2}} } \right)}}.
\end{equation}
and for the NIG this is simplified to,
\begin{equation}\label{appendixI:eq:nigchar}
{\phi _{NIG(\alpha ,\beta ,\delta ,\mu )}}(u) = {e^{\mu {\text{i}}u}}\frac{{{e^{\delta \sqrt {{\alpha ^2} - {\beta ^2}} }}}}{{{e^{\delta \sqrt {{\alpha ^2} - {{(\beta  + {\text{i}}u)}^2}} }}}}.
\end{equation}
In order to find the portfolio density in the case of the GO-GARCH (maGH/maNIG) model, the
characteristic function required for the inversion of the NIG density was already
used in \cite{Chen2007a} and given below,
\begin{equation}\label{appendixI:eq:nigcharsum}
{\phi _{port}}(u) = \exp \left\{ {{\text{i}}u\sum\limits_{j = 1}^d {{{\bar \mu }_j}}  + \sum\limits_{j = 1}^d {{{\bar \delta }_j}\left( {\sqrt {\bar \alpha _j^2 - \bar \beta _j^2}  - \sqrt {\bar \alpha _j^2 - {{({{\bar \beta }_j} + {\text{i}}u)}^2}} } \right)} } \right\}
\end{equation}
where $\bar \alpha_j$, $\bar \beta_j$, $\bar \delta_j$ and $\bar \mu_j$
represent the parameters scaled as described in the main text of the paper.
In the case of the GH characteristic function, this is a little more complicated
as it involves the evaluation of modified Bessel function of the
third kind with complex arguments.\footnote{The Bessel package
of \cite{Maechler2012} is used for this purpose.} Taking logs and summing,
\begin{eqnarray}\label{appendixI:eq:ghypcharsum}
{\phi _{port}}(u) = \exp\Biggl\{\text{i}u \sum\limits_{j = 1}^d \biggl({\bar \mu }_j+ \frac{\lambda _j}{2}\log{\left( {\bar \alpha _j^2 - \bar \beta _j^2} \right)} - \frac{\lambda _j}{2} \log{\left( \bar \alpha _j^2 - ({\bar \beta }_j + {\text{i}}u)^2 \right)} +  \nonumber \\
\log{\left( \bfK_{\lambda_j}\left( {\bar \delta }_j\sqrt{\bar \alpha _j^2 - ({\bar \beta }_j + \text{i}u)^2}\right) \right)} - \log{\left( \bfK_{\lambda _j}\left({\bar \delta}_j \sqrt{\bar \alpha _j^2 - \bar \beta _j^2} \right) \right)} \biggr) \Biggr\}
\end{eqnarray}
which is more than 30 times slower to evaluate than the equivalent NIG function
because of the Bessel function evaluations.
